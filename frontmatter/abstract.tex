%!TEX root = ../dissertation.tex
% the abstract

Flocking is a coordinated collective behavior that results from local sensing
between individual agents who have a tendency to orient towards each other.
Flocking is common amongst animal groups and could also be useful in robotic
swarms.
In the interest of learning how to control flocking behavior, several pieces
of recent work in the multiagent systems literature have explored the use of
influencing agents for guiding flocking agents to face a target direction.
However, the existing work in this domain has focused on simulation settings
of small areas with toroidal shapes.
In such settings, agent density is high, so interactions are common, and
flock formation occurs easily.
In our work, we study new environments with lower agent density, wherein
interactions are more rare.
We study the efficacy of placement strategies and influencing agent behaviors
drawn from the literature, and find that the behaviors that have been shown to
work well in high-density conditions tend to be much less effective in the
environments we introduce.
The source of this ineffectiveness is a tendency of influencing agents explored
in prior work to face directions intended for maximal influence that actually
separate the influencing agents from the flock.
We find that in low-density conditions maintaining a connection to the flock is
more important than rushing to orient towards the desired direction.

We use these insights to propose new influencing agent behaviors that overcome
the difficulties posed by our new environments.
The best influencing agents we identify act like normal members of the flock to
achieve positions that allow for control, and then exert their influence.  
We dub this strategy ``follow-then-influence.''

We also tackle this problem by using genetic programming to evolve ad hoc
behaviors for influencing agents.
We use a hand-constructed domain-specific language and evolve populations in
a small test environment, before testing the best candidates in larger
scenarios.
We find that the best genetic behaviors can do as well as our best
hand-designed algorithms, since they can strike a balance between quickly
turning their neighbors towards the goal direction while not losing influence
by flying away from their neighbors.
