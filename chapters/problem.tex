%!TEX root = ../dissertation.tex
\begin{savequote}[75mm]
This is some random quote to start off the chapter.
\qauthor{Firstname lastname}
\end{savequote}

\chapter{Problem Description}
\label{ch:problem}

\section{Flocking Model}
We use a simplified version of Reynold's Boid algorithm \cite{reynoldsmodel}
to model the flock.
In this simplified model, also proposed independently by Vicsek and
collaborators \cite{vicsek1995},
agents change their alignment at every step to be similar to the average
alignment of other agents in their neighborhood.
At each time step, each agent $a_i$ moves with constant speed $s=0.7$, has
orientation $\theta_i(t)$, and position $p_i(t) = (x_i(t), y_i(t))$.
At timestep $t$, $a_i$ updates its position based on its alignment:
$x_i(t) = x_i(t-1) + s\cos(\theta_i(t))$ and
$y_i(t) = y_i (t-1) - s\sin(\theta_i(t))$.
At the same time, the agents change their orientation based on the alignments
of neighboring agents.
Let the neighbors $N_i(t)$ be the set of agents at time $t$ that are within
neighborhood radius $r$ of $a_i$, not including $a_i$ itself.
At timestep $t$, each agent updates its orientation to be the average of their
neighbors' orientations:
\[\theta_i(t+1)=\theta_i(t)+\frac{1}{|N_i(t)|} \Sigma_{a_j \in N_i(t)}
\text{calcDiff}(\theta_j(t),\theta_i(t)),\]
where $\text{calcDiff}$ computes the difference between two angles, always
returning an angle between $0$ and $\pi$.

\section{Influencing Agents}
We can change flock dynamics by introducing influencing agents that we control.
We refer to non-influencing agents as Reynolds-Viscek agents.
We do not give the influencing agents any special control over Reynolds-Viscek
agents; we simply let the latter interact with influencing agents using the
same local sensing rules as with any other agent.
We also limit the influencing agents to have the same speed as Reynolds-Viscek
agents, both to help the influencing agents ``blend in" in real applications
and to conform to the Genter and Stone experiments.
We let the influencing agents have a sensing radius of twice the normal
neighborhood radius.
This allows for more complex behaviors like \textit{one step lookahead}
without requiring a global view from the influencing agent.
In some cases, the influencing agents can communicate with each other, but
they do not need global GPS.

We study influencing agents in three settings in this work.
First is the setting used in previous work by Genter and Stone; this setting
is a small toroidal $150\times150$ grid, with neighborhood radius $r=20$
\cite{genter2016facegoalfacecurrent, genter201612steplookahead, genterthesis}.
Because this grid is relatively small, and the neighborhood is relatively
large, this simulates a very high-density flocking setting.
In this setting, convergence is rapid, so our primary metric of interest is
time to convergence.
We call this setting the \textit{small} setting.

In order to study low-density dynamics, we introduce two new settings that are
more adverse to flock formation; we call these new settings the \textit{large}
setting and the \textit{herd} setting.
In both these settings, we set the neighborhood radius to $r=10$.

In the \textit{large} setting, non-influencing agents are randomly placed in a
toroidal $1000\times1000$ grid with random initial orientations.
The larger grid size results in lower agent density; as a result, agents start
out much farther away from other agents' neighborhoods, and interactions are
much rarer.
However, since the simulation space remains toroidal, convergence to a single
flock is still provably guaranteed, so we are primarily interested in
studying the length of time to convergence in this case
\cite{jad2003convergence}.
In the \textit{herd} setting, non-influencing agents are placed randomly in a
circle of radius 500 whose origin lies at the center of a $5000\times5000$
non-toroidal grid.
When agents reach the edge of the grid, they simply keep going; in this way,
agents can get ``lost" from the rest of the flock.
As a result, convergence to a single flock is not guaranteed.
Therefore, we are interested in studying how well influencing agents can keep
the non-influencing agents from getting lost.
