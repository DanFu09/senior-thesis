%!TEX root = ../dissertation.tex
\chapter{The Coordinated Behavior}
\label{ch:AppendixB}

We present a more detailed explanation of Genter and Stone's Coordinated
Behavior \cite{genter201612steplookahead}.

The Coordinated Behavior aims to pair up influencing agents that share a large
number of Reynolds-Vicsek agents in their neighborhoods and have the pairs
jointly run a \textit{one step lookahead} algorithm.
The only difficult portion in this behavior is forming the pairs; once they are
formed, \textit{one step lookahead} can be used without modification.
Absent any optimizations, this is an $O(mn^2)$ [NOTE MAKE THIS A BIG O]
algorithm, where $n$ is the number of influencing agents, and $m$ is the number
of Reynolds-Vicsek agents; the $n^2$ factor comes from the fact that there are
only $O(n^2)$ possible pairings, and the $m$ factor reflects the worst case,
where all the agents are clustered in a very tight space.
The Coordinated algorithm optimizes this search slightly by only considering
influencing agents that can share Reynolds-Vicsek agents as neighbors, i.e.
only those influencing agent pairs which are less than twice the neighborhood
radius away from each other.
Then the algorithm simply iterates through all pairings and chooses the one
that maximizes the number of Reynolds-Vicsek agents that are in the
intersection of the neighborhoods of two influencing agents.
Although simple implementations of this algorithm may require a global
controller to implement, this algorithm can theoretically be done in a
completely decentralized manner.
