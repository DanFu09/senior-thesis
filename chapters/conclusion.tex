%!TEX root = ../dissertation.tex
\chapter{Conclusion}
\label{ch:conclusion}

We have studied the problem of controlling flocks using influencing agents
under two new, more adversarial environments with lower agent density, and
have introduced some novel behaviors for these settings.
Besides these new algorithms, we have found that, in low-density
environments, it is more important for influencing agents to
\textit{maintain influence} than it is for them to rapidly turn their
neighbors towards the correct destination.
Consequently, earlier results from smaller simulation environments often do
not hold in the environments we introduce.
We have found that a multistage approach that first embeds influencing
agents in small flocks before attempting to steer these flocks to the goal
direction can be effective in addressing some of these shortcomings.

We have also presented an evolutionary algorithm that evolves new genetic
algorithms for these influencing agents.
We evolved on a genome based on a hand-constructed domain-specific language
and found that many of the evolutionary algorithms incorporated non-trivial
insights about the best way to be successful in these settings.
However, we also found that the evolved behaviors could be difficult to parse.

\section*{Future Work}
There are a number of promising future directions that we could take from this
work.
We could explore the design space of genetic algorithms more deeply by
evolving behaviors over different genomes, such as neural networks.
We could also co-evolve different aspects of influencing agent behavior to help
agents co-ordinate in previously unexpected ways; for example, we could
introduce some form of communication between influencing agents, or co-evolve
behaviors with placement strategies.

Future work could also explore the design space of placement strategies and
agent behaviors by applying machine learning techniques to this problem;
reinforcement learning seems to be a particularly intriguing way to develop new
behaviors for influencing agents.
We would also like to explore the question of how to aggregate small flocks
into one larger flock.
Almost all of our behaviors result in multiple small flocks clustered around
small numbers of influencing agents; these flocks have converged in the sense
that they are all facing the same direction, but they remain disconnected from
each other.
An interesting challenge would be to develop algorithms to merge small flocks
that start out with the same orientation, while maintaining flock composition.
For this challenge, a successful algorithm for an influencing agent must change
the direction of the flock without losing individual Reynolds-Vicsek agents on
the edge of the flock.

Another direction would be to change the flocking dynamics and see how the
various algorithms perform in that case.
Our work has focused on a simplified version of the Reynolds flocking model and
has only studied alignment dynamics.
The original Reynolds model also includes avoidance and cohesion dynamics.
These dynamics are important in real flocks, and in the future, the effects of
including these dynamics should be explored.

%\section{Radhika Conclusion}
%In this paper, we have presented an evolutionary algorithm that evolves
%ad hoc algorithms for influencing agents to control flocks of birds.
%We evolved a genome based on a hand-constructed domain-specific language
%and found that many of the evolutionary algorithms incorporated non-trivial
%insights about the best way to be successful in a wide range of settings.
%However, we also found that the evolved behaviors could be difficult to 
%parse and somewhat unpredictable, especially when used as parts of more
%complicated behaviors.
%
%In the future, we plan on conducting a more thorough analysis of these
%genetic behaviors in a wider range of evaluation settings.
%We also wish to explore the design space of genetic algorithms more deeply
%by evolving behaviors over different genomes, such as neural networks.
%Another promising direction of research is to co-evolve different aspects
%of influencing agent behavior to help agents co-ordinate in previously
%unexpected ways.
%Finally, we could also explore the design space in a different way by
%taking techniques from reinforcement learning.

% \section {Future work}
% \label{sec:future}
% The existing literature studying influencing agents in flocks has almost
% exclusively focused on a simplified version of the Reynolds flocking model.
% In short, the simplified model assumes that the only inter-agent dynamics in
% play are alignment dynamics - agents tend to align their velocity vectors with
% neighboring agents.
% The original Reynolds model also includes avoidance and cohesion dynamics -
% agents will try not to collide with other agents, and agents will tend towards
% the center position of a group.
% These dynamics are important in real flocks, and in the future, the effects of
% initializing the cohesion and avoidance assumptions in Reynolds' flocking model
% should be explored.
% We briefly tested flocks with cohesion and avoidance dynamics throughout the
% course of our work and observed a major change in flocking dynamics when we did
% so.
% In particular, the introduction of cohesion and avoidance dynamics allows the
% system to be stable in an oscillatory state.
% This discourages flock formation and makes the problem significantly more
% difficult.
% Careful analysis of the effects of cohesion and avoidance is required to
% develop new metrics and influencing agent behaviors.

% Furthermore, in \textit{herd}, we believe the \textit{multicircle transition} 
% approach can be improved to more effectively steer flocks towards outer circles 
% without losing agents in the process.
% Similarly, further study of how to change the direction of an existing flock
% can improve the efficacy of our \textit{multistep} behavior in the \textit{large};
% perhaps a ``nudge, then follow" approach would be effective.
