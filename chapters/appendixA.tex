%!TEX root = ../dissertation.tex
\chapter{Other Behaviors Explored}
\label{AppendixA}

In this Appendix, we present some extra behaviors that we explored but chose
not to present in the main text.
These are primarily variations on the main behaviors presented throughout the
text, and come of a few flavors: some are variations on the behaviors presented
for the \textit{large} setting that modulate influencing agent behavior when
there are no Reynolds-Vicsek agents present; some of these same variations can
also be applied to the net behaviors in the \textit{herd} setting.
Other variations are minor variations on the stationary behaviors in the
\textit{herd} setting, in the same vein as the variations of the
\textit{multistep} algorithm that we explored in the main text.

To understand where all these variations are coming from and to help organize
their presentation, it is useful to think of behaviors as having \textit{local}
and \textit{global} components.
The local component takes a desired orientation as an argument and dictates
which direction an influencing agent will face to try to influence its
neighbors towards the goal orientation.
The global component, on the other hand, dictates the desired orientation to
feed to the local component and how the influencing agents might coordinate to
decide on a desired orientation.
For example, one behavior might be for the influencing agents to circle around
the center of a grid and get the Reynolds-Vicsek agents to circle with them.
The global component would compute the desired orientation necessary at each
step to maintain the circling behavior, and the local component would compute
which exact direction to face to get the Reynolds-Vicsek agents to face the
desired orientation at that step.

Presented this way, the \textit{face}, \textit{offset momentum}, \textit{one
step lookahead}, \textit{coordinated}, and \textit{Couzin} behaviors can all
be candidates for local components of more complex algorithms, and so can all
the genetic behaviors.
The \textit{multistep}, \textit{circle}, \textit{polygon}, and
\textit{multicircle} algorithms, on the other hand, can be thought of as global
components.
We have already explored the pairings that arise when matching the
aforementioned local components with the \textit{multistep} behavior; these are
the variations presented in \S\ref{ch:evaluation}.
However, we can imagine an entire array of new behaviors generated by pairing
those local components with the various stationary algorithms.
Given this framework, we can also come up with some new global components for
behaviors.

\section{Non-Circular Global Behaviors}
We'll start with non-circular global behaviors.
These can be applied to the \textit{large} setting or the net case of the
\textit{herd} setting.
The simplest global behavior is for the local component to always be fed the
pre-determined goal direction.
This results in a set of ehaviors that are exactly the same as the behaviors
presented earlier in this text.
We can refer to this global behavior as the \textit{direct} global behavior.

We can come up with a new global behavior if we consider what to do with
influencing agents when there are no Reynolds-Vicsek agents in their
neighborhood.
Under the \textit{direct} behavior, the influencing agents will simply continue
traveling in the goal direction.
This can be problematic; consider, for example, what happens if influencing
agents almost succeed in turning a flock towards the goal direction, but fall
slightly short.
Then the flock will be traveling on a vector almost parallel to the goal
direction, and so will the influencing agents; this can make it difficult for
the influencing agents to collide with the flock, especially in a toroidal
world.
We can solve this problem by having the influencing agents choose a random
direction as soon as there are no Reynolds-Vicsek agents in their neighborhood.
One way of looking at this behavior is that it has influencing agents influence
their neighbors if they have any, or choose a random direction otherwise.
We can call this behavior the \textit{random} global behavior.
Like the \textit{direct} behavior, it can be applied to both the \textit{large}
setting and the net case of the \textit{herd} setting.

What else can we do with the net case of the \textit{herd} setting?
Perhaps we can try for an adaptation of the \textit{multistep} global behavior.
As we stated in the main text, this is difficult because in a non-toroidal
setting, the total number of Reynolds-Vicsek agents under influencing agent
control may never exceed any particular threshold $T$.
We may never be able to successfully produce a ``follow-then-influence"
approach in this setting, but perhaps we can replicate a different aspect of
the approach by chaining two other behaviors together.
One simple example would be for influencing agents to use the \textit{circle}
global behavior until coming into contact with Reynolds-Vicsek agents, after
which they adopt the \textit{direct} global behavior.
We call this the \textit{circle-net} global behavior.

So how do these behaviors do?
Across the board, they are worse than the \textit{direct} global behavior.
When paired with the hand-designed local behaviors, the \textit{random} global
behavior performs about on par with the \textit{direct} global behavior, but
pairing \textit{random} with certain genetic local behaviors produces wildly
varying results.
Some combinations of \textit{random} with genetic behaviors start performing
an order of magnitude worse than \textit{direct} with \textit{face}.
In the net case of the \textit{herd} setting, the \textit{random} behavior
performs very poorly; when there is no toroidal world to give the influencing
agents a chance to run into flocks again, the random behavior at the beginning
simply reduces the number of useful influencing agents.
The new \textit{circle-net} behavior does not perform as badly as
\textit{random}, but is still not as good as the \textit{direct} behavior.

\section{Circular Global Behaviors}
Next, we move on to new behaviors for the stationary case of the \textit{herd}
setting.
First, note that we can pair any of the local behaviors that we already have
with the global behaviors for the stationary case of the \textit{herd} setting;
the circle, polygon, and multicircle behaviors are all just specifications for
changing a goal direction over time.

But we can also come up with some more complex variations on the multicircle
behavior.
An obvious one is simply to change the final radius, but a more intriguing
one is to have a transition period between the following stage and the final
circle.
We call the following behavior \textit{multicircle-transition.}

This behavior is parameterized by three variables, a placement radius $r_P$, a
threshold radius $r_T$ and a final radius $r_F$.
Agents start by tracing a circle around the origin of radius $r_P$; $r_P$ is
defined by the agent's placement under the placement strategy.
Once the agents come into contact with Reynolds-Vicsek agents, they follow the
Reynolds-Vicsek agents until they reach a threshold radius $r_T$.
From there, they trace a transition path until they are a radius $r_F$ from the
origin, where they settle into a final trajectory tracing the circle of radius
$r_F$ around the origin.

We compute the transition path by finding a circle that is tangent to both the
influencing agent's velocity vector when it is at radius $r_T$ and the final
circle of radius $r_F$.
This problem reduces to one of geometry; we wish to find a point $O'$ around
which to trace a transition circle.
For the rest of this computation, assume that the influencing agent is at the
threshold radius $r_T$.

We can reduce the set of candidate points by taking the line orthogonal to the
agent's velocity vector that intersects the agent's velocity vector at the
agent's position.
Consider a point $O'$ on this line, and let $r'$ be the distance of
$O'$ from the agent.
Then, the circle of radius $r'$ centered around $O'$ is tangent to
the agent's velocity vector at the agent's position.
Since we know the form of the line that $O'$ is on, we simply need to
determine a proper radius $r'$.

We have one more condition: that the transition circle be tangent to the larger
circle.
For this to be the case, we must have that the distance from $O'$ to the
larger circle be $r'$.
Let $r_{O}$ be the distance from $O'$ to the origin of the larger circle.
Then we have $r' = r_F - r_{O}$.
This gives us the proper value of $r'$, and two candidates for $O'$.
We choose the point that causes the agent to enter the larger circle clockwise,
to prevent collisions on the larger circle.
During the actual simulations, we calculate this point numerically, walking
along both sides of the orthogonal line until finding the right candidate.

So how do these behaviors do?
When paired with the \textit{offset momentum}, \textit{one step lookahead}, or
\textit{coordinated} behaviors, they fail spectacularly.
In all cases, each one of these global behaviors fails to keep any
Reynolds-Vicsek agents under influencing control after $15000$ steps.
These local behaviors are so bad at maintaining influence that they simply
can't induce circling behavior of any kind.
When paired with the \textit{face} local behavior, none of the other global
behaviors perform as well as the variation of the \textit{multicircle} behavior
presented in the main text.

TALK ABOUT COUZIN A BIT 
